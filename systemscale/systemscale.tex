\chapter{System-Scale Data Locality Management}
\label{ch:systemscale}
%% EJ : Here is what I remember we decided at the meeting: 
%% There are 3 points : 
%% 1) Topology-mapping
%% 2) storage and data
%% 3) Resource management and batch scheduler


%% This section should contain an overview of the different problematics and a
%% description of existing solutions and open problems. 





SCOPE:

\begin{itemize}
\item topology mapping
\item storage and data
\item Resource management and batch scheduler
\end{itemize}

KEY POINTS:
\begin{itemize}
\item Whatever the application is written and optimize, the way it is executed
  has a huge impact on its performance
 \item Mapping of application to resources is important even in the case of
   contiguous allocation. But joint allocation and mapping  (application
   expresses its need and the batch scheduler tries to fulfill it) provides
   better optimization
\item there is a need of models of the machine at different scales (cache, node,
memory, network, etc.) Such models are necessary to design algorithms to enhance
performance
  \item Usually, each part of the application ecosystem (storage, runtime, batch
    scheduler, etc.) act independently. There is a need for joint decisions both
    vertically (from node to storage) and horizontally (e.g. between compute
    nodes)
\item need of model of application (same reason as above)
\item \ldots
\end{itemize}

RELATED WORK: 

\begin{itemize}
\item Topology mapping: TreeMatch, Libtopomap, MPIPP? scotch, metis  
\item Storage: parallel file system (???)
\item resource management: queuing systems 'SLURM, LSF, PBS, etc.)
\item machine models: HWLOC, NETLOC, etc. 
\item building app model: profilers (scalasca), compilers, runtime monitoring (openMPI), etc. 
\end{itemize}

CHALLENGES:
\begin{itemize}
\item scalability and management of the hierarchy
\item globality (tackle the problem system wide)
\item interaction between different layers
\item \ldots
\end{itemize}


\section{Introduction}
%% An idea of the presentation 
 

Parallel computers are becoming more and more complex. 
Indeed, they feature hundreds of thousand of cores, a deep memory hierarchy with 
several cache layers, non-uniform memory access with  several levels of memory (flash,
non-volatile, RAM, etc), elaborate topologies (both at the
shared-memory level and at the distributed-memory one with state-of-the-art
interconnection network) and advanced parallel storage systems and resource
management tools.

Nowadays it is already more challenging to write an application that efficiently
access its data than efficiently process the data. In other words bytes 
feature more issues issues than flops. As it is expected that the memory
per core will decrease in the coming years this problematic will become even
more important. 

To address the data locality issue system-wide, one approach consists in
working on the application ecosystem and how the application executed (interaction
with the batch-scheduler and the storage, optimization on the way the
application uses the resources, interaction with the topology, etc.). 

More precisely, the important issues lies in the strong need for new models and algorithms, new 
mechanisms and tools for improving (1) topology-aware data accesses, (2) data 
movements across the various software layers, (3) data locality and transfers for 
applications. 

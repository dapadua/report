\chapter{Data Locality in Task Models}
\label{ch:taskmodels}


%Define your Area
%  Create definition of your research area
%  Describe key concepts that define your area or that were uncovered during the course of conversation
%  A few examples of work in that area (can refer back to talks on website, but no need to recount entire talk)

%Findings:
%■ Describe points/observations/discoveries/challenges/issues uncovered in the session
 %‚óè Distill into summary (major discoveries)
 %‚óè can refer back to presentations for details
 %‚óè Can also use data from panel discussions
%■ Identify areas of agreement
 %‚óè Common approaches
 %‚óè Common concerns
%■ Identify areas of disagreement
 %‚óè what is the substantive cause of the disagreement (document)
 %‚óè What metrics/information/research are needed to compare/resolve
%■ Identify Gaps
 %‚óè What is missing?

%Recommendations
%■ Opportunities for standardization of mature technologies where the is substantial agreement or commonality
 %‚óè Have we met the necessary conditions for standardization (is the area well enough understood, are the elements of existing implementations sufficiently similar, are the benefits clearly demonstrated, is there a user community?)
 %● What should we standardize? ( Low­hanging fruit )
 %‚óè How can we influence standards committees? (e.g. C++17standards committee?)
%■ Define research agenda for new ideas or areas where there is insufficient information to choose a final implementation ( What areas need more research?)
 %‚óè identify research thrust
 %‚óè what are the opportunities
 %‚óè what needs to be done
 %‚óè What needs to be prioritized?
 %‚óè What resources would be required (estimate size/complexity ofthe problem if you can)
%■ How do we create a user community? (bonus question)


\section{Introduction}

SCOPE (\textbf{Jesus})
	\begin{itemize}
		\item Design of interfaces to express
		      locality in a task-based programming model
		\item Terminology: ``task-based'', ``interfaces'' (mostly to specify the levels of the interfaces)
		\item task-based parallelism is the way to go for many applications (MPI+Task)
		\item Fast prototyping / productivity
		\item Redesigning numerical algorithms to express locality
		and how to express it (nested parallelism, divide and conquer, tree structure)
		\item Interface betw app/rt/hw to support locality
		\item Specific performance/debugging tools for task-based applications
		\item Compiler technology RT Vs Library RT
	\end{itemize}
	
	
	
\section{Key Points}


DISCOVERIES/CHALLENGES
	\begin{itemize}
		\item Task granularity (flexibility) (\textbf{Miquel}): 
		Large tasks (i.e. coarse-grained tasks) can result in load imbalance at synchronization points. Reducing task sizes is a common method
		to improve load balance.  On the other hand, the granularity of tasks directly influences scheduling overheads. Too fine-grained tasks
		increase the amount of time spent inside the runtime performing tasks such as scheduling and work stealing.  Task granularity impacts
		locality because larger tasks also have bigger memory footprints. Task sizes are not only a trade-off between parallelism and runtime
		overheads, but should also be set in order to efficiently exploit the memory hierarchy. Last level shared caches provide larger
		storage that can be exploited to improve performance and reuse. The optimal granularity depends on data sharing between tasks:
		\begin{itemize}
			\item In the absence of data sharing one should target a task footprint of $\frac{\mbox{SharedCacheSize}}{\mbox{Ncores}}$. 
			\item In the extreme case, if all data is shared, the target task footprint can become $\mbox{SharedCacheSize}$. In general it
				will lie somewhere in between these two cases.
		\end{itemize} 
		Finding the best granularity is a complex problem as all three metrics to be optimized (overheads, load balance, memory hierarchy) are
		coupled.  Since the optimum configuration may be a data input dependent problem, auto-tuning techniques should be explored.  Enabling
		autotuning involves programmer effort to find ways to partition data sets in a parametric way, allowing to tune the task size. This is
		also a problem dependent issue. To enable parametric granularity control, program and data decomposition should preferably be
		performed following a top-down approach rather than bottom-up.
		\item Scheduling overhead (\textbf{Miquel}):
			Scheduling overhead refers to the time required by the runtime to generate the execution plan of a DAG computation. It
			includes everything that happens between two execution units except for waiting time (synchronization).  Depending on the
			tasking semantics this involves different amount of work:
			\begin{itemize}
				\item  {\emph Task-parallel}: In task-parallel runtimes, spawned tasks are ready to be executed (cite: Cilk, TBB
					task\_groups, OpenMP 3.0 Tasks). In this scenario scheduling involves fetching a task from the ready queue(s)
					or running the work-stealing loop if there are no ready tasks in the queues. This scheme has low overhead but
					supports mainly divide-and-conquer and parallel loop style of computation.
				\item  {\emph Task-dataflow}: In dataflow schemes (cite: OmpSs, OpenMP 4.0 tasks, TBB dependencies) there are two
					scheduling levels: 1) resolving dependencies to find ready tasks and 2) scheduling ready tasks to
					workers. The latter is in general performed as in task-parallel schemes. 
			\end{itemize} 
				Because of these differences, task-parallel runtimes tend to have smaller scheduling overheads compared to
				task-dataflow runtimes. It is an open research problem to reduce the overheads of task-dataflow schedulers in order to
				efficiently support tasks of finer granularity. Some researchers propose to replace the dependency-tracking scheduler
				with a ticket-based approach (cite: SWAN). Other groups propose to implement the parts of the scheduler in hardware,
				such as the dependency tracking mechanism (cite: HTSS, Nexus++) or the ready queues and task scheduler (cite:
				Carbon).
		\item Efficient debugging/tracing mechanisms/tools (\textbf{Jesus})
		\item Nested parallelism (recursive formulation) (\textbf{Hatem})
		\item Work stealing (\textbf{Miquel}):
			Work stealing is how idle cores obtain work in order to balance the load across processors. When work is stolen, the working
			set of the task is also "stolen". Thus a work steal operation usually results in a data migration.  The impact and
			optimization of work stealing depends on the particular case:
				\begin{itemize}
					\item Obviously, work stealing can only be optimized for locality if tasks feature sufficient
						internal locality and are to an extent memory bound. If tasks have very bad locality or the miss rate
						to main memory is already negligible, then work stealing has little effect beyond the overhead of the
						stealing itself. 
					\item When tasks have locality of reference, things become interesting. If sets of tasks can constructively
						share the cache then limiting the work stealing to the tasks within the set will limit the working set
						migration to the private caches (shared caches will not be affected).  Parallel-depth-first schedulers
						attempt to constructively share the cache by scheduling the oldest sequential task in the set of cores
						sharing the cache and only resorting to global work stealing when the task queues become empty.
					\item In general we want to minimize the number of work steals. This works well if the application can be
						rapidly partitioned into almost equivalent sets of work that can then proceed independently. This
						works well for divide-and-conquer parallelism, but for more general approaches such as loop-style
						parallelism (i.e., tasks generated inside a {\tt for} loop) using work stealing to keep all cores busy
						is generally not efficient. This is because distributing $N$ tasks generated by a $N$-iteration loop
						will require exactly $N$ work steals. If N is larger than the number of processors, then a worksharing
						partitioning of the loop can be a more efficient method.
					\item A big challenge is for the runtime to learn the hierarchical data properties of an application and
						exploit them to generate efficient schedules. Classical random work stealing (e.g.  Cilk-like) does
						not do any attempt to exploit this. Socket-aware policies exist (eg, Qthread) that perform
						hierarchical work stealing: 1) first among cores in a socket and 2) then among sockets.  Some
						programming models expose an API that allows programmers specify on which NUMA node/socket a
						collection of tasks should be executed (cite: OmpSs). Configurable work stealers which can be
						customized with {\it scheduling hints} have also been developed (cite: Pheet/PPoPP'13). Finally, a
						more extreme option is to allow the application programmer to attach a custom work stealing function
						to his application (e.g., MassiveThreads/ROSS13). How to effectively specify this information in a
						programmer-friendly way is an open topic of research.
				\end{itemize}
		\item Scheduling priority, DAG critical path  (\textbf{Romain})
		\item Socket-aware scheduling (\textbf{Hatem})
		\item Detection of overlapped memory-region (\textbf{Jesus})
		\item API Standardization (\textbf{Romain})
		\item DAG Composition (\textbf{Hatem})
		\item Handling data locality in presence of co-processors and accelerators (\textbf{Jesus})
	\end{itemize}


\section{The State-of-the-Art}

A few examples of work in that area  (\textbf{Hatem})
\begin{itemize}
	\item OmpSs
	\item Charm++
	\item QUARK
	\item StarPU
	\item PaRSEC
	\item SuperMatrix
	\item ParalleX
	\item OCR
\end{itemize}

APPLICATION DOMAINS  (\textbf{Hatem})
		\begin{itemize}
			\item Dense linear algebra
			\item Computational astronomy
			\item Uncertainty Quantification
			\item Seismic
			\item Weather/Climate modeling
			\item Krylov-based solvers
			\item Nbody problems
		\end{itemize}

\section{Discussions}

AREAS OF AGREEMENT (\textbf{Romain})
          \begin{itemize}
          \item{
              The performance and energy characteristics of today's
hardware are difficult to predict; the unpredictability of a hit or
miss in a cache, the variable latencies introduced by branch
mis-predictions, etc.\ are only some of the factors that contribute to
a very dynamic hardware behavior. This trend will only accelerate with
future hardware as near threshold voltage (NTV) and aggressive energy
management increase the level of hardware dynamism. [TODO: Maybe
phrase it to say that the assembly interface does not guarantee
anything in terms of performance and/or energy].

              In light of this, toolchains will become more and more
unable to statically partition and schedule code to efficiently
utilize future parallel resources. Hardware characteristics will vary
dynamically and we therefore cannot do without a dose of dynamism in
the programming model: dynamic task based programming models need to
be a part of the solution.
            }
          \item{
              Task base programming models, however, rely on the
computation and data being split into chunks (``tasks'' for the
computation which implies a size of the data these tasks operate
on). The size of these chunks (the granularity) is difficult to
determine as it needs to balance the overheads of the task-based
system with the need to expose sufficient parallelism to fully occupy
the ever increasing number of parallel resources. A static granularity
will be sub-optimal for future machines [TODO: Do I need to add more
to this]
            }
          \end{itemize}
% OLD STUFF
		  % \begin{itemize}
		  % \item Static partitioning does not work for everything; we need to
		  %     have a dynamic component.
		  % \item A perfectly predictable hardware does not exist and, in the
		  %     future will be even more far-fetched; we therefore cannot get
		  %     away from some level of dynamism and task-based
		  %     programming-models can make better use of this dynamism.
		  % \item Granularity is a key issue that needs to be addressed with
		  %     respect to locality. There is agreement that a static
		  %     partitioning/granularity will not work for future
		  %     machines. Granularity needs to balance overheads and exposing
		  %     parallelism.
		  % \end{itemize}



AREAS OF DISAGREEMENT (\textbf{Romain})
          \begin{itemize}
            \item{
                There is agreement on the fact that a standardization
needs to happen for task based programming model but there is
disagreement as to the level at which this standardization needs to
happen. One option is to standardize the APIs at the runtime level in
a way similar to the Open Community Runtime (OCR). Another option is
to standardize what can be expressed by the user at the programming
model level [TODO: Jesus, I would like to understand your position a
bit more on this so I can clarify things a bit more].
              }
            \item{
                Another area of disagreement is how best to deal with
 the expression of granularity; specifically, is it better for the
 programmer to break-down tasks and data and have a runtime system
 re-assemble them if needed or is it preferable to have the programmer
express coarse grained tasks and data and allow the runtime system to
break them down further. The latter approach has been used
successfully in runtimes targeted to problems that are recursively
divisible. The former approach would require a sort of ``recipe'' for
the runtime to be able to stich smaller tasks or chunks of data into
larger ones. There is debate as to which approach is simpler and more
likely to yield positive results.
              }
            \end{itemize}




\section{Research Plan}


Identify Gaps / What is missing  (\textbf{Miquel})
	    \begin{itemize}
		\item Better understanding of the performance features of task-parallel
		systems. Need to convince developers of the value of these systems.
		\begin{itemize}
		  \item Better performance tools. Better understanding of the impact of locality
		  \item Case studies that show performance benefit in certain conditions (eg,
		    in the presence of faults?). Focus on data aspects, i.e. show that
		    a good management can be done even when there are a lot of work steals.
	      \end{itemize}
	  \end{itemize}

































































% HIGHLIGHTS
% 
% \begin{itemize}
% 
% 
% 	\item SCOPE (\textbf{Jesus})
% 	\begin{itemize}
% 		\item Design of interfaces to express
% 		      locality in a task-based programming model
% 		\item Terminology: ``task-based'', ``interfaces'' (mostly to specify the levels of the interfaces)
% 		\item task-based parallelism is the way to go for many applications (MPI+Task)
% 		\item Fast prototyping / productivity
% 		\item Redesigning numerical algorithms to express locality
% 		and how to express it (nested parallelism, divide and conquer, tree structure)
% 		\item Interface betw app/rt/hw to support locality
% 		\item Specific performance/debugging tools for task-based applications
% 		\item Compiler technology RT Vs Library RT
% 	\end{itemize}
% 
% 
% 
% 	\item RELATED WORK: a few examples of work in that area  (\textbf{Hatem})
% 	\begin{itemize}
% 		\item OmpSs
% 		\item Charm++
% 		\item QUARK
% 		\item StarPU
% 		\item PaRSEC
% 		\item SuperMatrix
% 		\item ParalleX
% 		\item OCR
% 	\end{itemize}
% 
% 
% 	\item DISCOVERIES/CHALLENGES
% 	\begin{itemize}
% 		\item Task granularity (flexibility) (\textbf{Miquel}): 
% 		Large tasks (i.e. coarse-grained tasks) can result in load imbalance at synchronization points. Reducing task sizes is a common method
% 		to improve load balance.  On the other hand, the granularity of tasks directly influences scheduling overheads. Too fine-grained tasks
% 		increase the amount of time spent inside the runtime performing tasks such as scheduling and work stealing.  Task granularity impacts
% 		locality because larger tasks also have bigger memory footprints. Task sizes are not only a trade-off between parallelism and runtime
% 		overheads, but should also be set in order to efficiently exploit the memory hierarchy. Last level shared caches provide larger
% 		storage that can be exploited to improve performance and reuse. The optimal granularity depends on data sharing between tasks:
% 		\begin{itemize}
% 			\item In the absence of data sharing one should target a task footprint of $\frac{\mbox{SharedCacheSize}}{\mbox{Ncores}}$. 
% 			\item In the extreme case, if all data is shared, the target task footprint can become $\mbox{SharedCacheSize}$. In general it
% 				will lie somewhere in between these two cases.
% 		\end{itemize} 
% 		Finding the best granularity is a complex problem as all three metrics to be optimized (overheads, load balance, memory hierarchy) are
% 		coupled.  Since the optimum configuration may be a data input dependent problem, auto-tuning techniques should be explored.  Enabling
% 		autotuning involves programmer effort to find ways to partition data sets in a parametric way, allowing to tune the task size. This is
% 		also a problem dependent issue. To enable parametric granularity control, program and data decomposition should preferably be
% 		performed following a top-down approach rather than bottom-up.
% 		\item Scheduling overhead (\textbf{Miquel}):
% 			Scheduling overhead refers to the time required by the runtime to generate the execution plan of a DAG computation. It
% 			includes everything that happens between two execution units except for waiting time (synchronization).  Depending on the
% 			tasking semantics this involves different amount of work:
% 			\begin{itemize}
% 				\item  {\emph Task-parallel}: In task-parallel runtimes, spawned tasks are ready to be executed (cite: Cilk, TBB
% 					task\_groups, OpenMP 3.0 Tasks). In this scenario scheduling involves fetching a task from the ready queue(s)
% 					or running the work-stealing loop if there are no ready tasks in the queues. This scheme has low overhead but
% 					supports mainly divide-and-conquer and parallel loop style of computation.
% 				\item  {\emph Task-dataflow}: In dataflow schemes (cite: OmpSs, OpenMP 4.0 tasks, TBB dependencies) there are two
% 					scheduling levels: 1) resolving dependencies to find ready tasks and 2) scheduling ready tasks to
% 					workers. The latter is in general performed as in task-parallel schemes. 
% 			\end{itemize} 
% 				Because of these differences, task-parallel runtimes tend to have smaller scheduling overheads compared to
% 				task-dataflow runtimes. It is an open research problem to reduce the overheads of task-dataflow schedulers in order to
% 				efficiently support tasks of finer granularity. Some researchers propose to replace the dependency-tracking scheduler
% 				with a ticket-based approach (cite: SWAN). Other groups propose to implement the parts of the scheduler in hardware,
% 				such as the dependency tracking mechanism (cite: HTSS, Nexus++) or the ready queues and task scheduler (cite:
% 				Carbon).
% 		\item Efficient debugging/tracing mechanisms/tools (\textbf{Jesus})
% 		\item Nested parallelism (recursive formulation) (\textbf{Hatem})
% 		\item Work stealing (\textbf{Miquel}):
% 			Work stealing is how idle cores obtain work in order to balance the load across processors. When work is stolen, the working
% 			set of the task is also "stolen". Thus a work steal operation usually results in a data migration.  The impact and
% 			optimization of work stealing depends on the particular case:
% 				\begin{itemize}
% 					\item Obviously, work stealing can only be optimized for locality if tasks feature sufficient
% 						internal locality and are to an extent memory bound. If tasks have very bad locality or the miss rate
% 						to main memory is already negligible, then work stealing has little effect beyond the overhead of the
% 						stealing itself. 
% 					\item When tasks have locality of reference, things become interesting. If sets of tasks can constructively
% 						share the cache then limiting the work stealing to the tasks within the set will limit the working set
% 						migration to the private caches (shared caches will not be affected).  Parallel-depth-first schedulers
% 						attempt to constructively share the cache by scheduling the oldest sequential task in the set of cores
% 						sharing the cache and only resorting to global work stealing when the task queues become empty.
% 					\item In general we want to minimize the number of work steals. This works well if the application can be
% 						rapidly partitioned into almost equivalent sets of work that can then proceed independently. This
% 						works well for divide-and-conquer parallelism, but for more general approaches such as loop-style
% 						parallelism (i.e., tasks generated inside a {\tt for} loop) using work stealing to keep all cores busy
% 						is generally not efficient. This is because distributing $N$ tasks generated by a $N$-iteration loop
% 						will require exactly $N$ work steals. If N is larger than the number of processors, then a worksharing
% 						partitioning of the loop can be a more efficient method.
% 					\item A big challenge is for the runtime to learn the hierarchical data properties of an application and
% 						exploit them to generate efficient schedules. Classical random work stealing (e.g.  Cilk-like) does
% 						not do any attempt to exploit this. Socket-aware policies exist (eg, Qthread) that perform
% 						hierarchical work stealing: 1) first among cores in a socket and 2) then among sockets.  Some
% 						programming models expose an API that allows programmers specify on which NUMA node/socket a
% 						collection of tasks should be executed (cite: OmpSs). Configurable work stealers which can be
% 						customized with {\it scheduling hints} have also been developed (cite: Pheet/PPoPP'13). Finally, a
% 						more extreme option is to allow the application programmer to attach a custom work stealing function
% 						to his application (e.g., MassiveThreads/ROSS13). How to effectively specify this information in a
% 						programmer-friendly way is an open topic of research.
% 				\end{itemize}
% 		\item Scheduling priority, DAG critical path  (\textbf{Romain})
% 		\item Socket-aware scheduling (\textbf{Hatem})
% 		\item Detection of overlapped memory-region (\textbf{Jesus})
% 		\item API Standardization (\textbf{Romain})
% 		\item DAG Composition (\textbf{Hatem})
% 		\item Handling data locality in presence of co-processors and accelerators (\textbf{Jesus})
% 	\end{itemize}
% 
% 
% 
% 	\item AREAS OF AGREEMENT (\textbf{Romain})
%           \begin{itemize}
%           \item{
%               The performance and energy characteristics of today's
% hardware are difficult to predict; the unpredictability of a hit or
% miss in a cache, the variable latencies introduced by branch
% mis-predictions, etc.\ are only some of the factors that contribute to
% a very dynamic hardware behavior. This trend will only accelerate with
% future hardware as near threshold voltage (NTV) and aggressive energy
% management increase the level of hardware dynamism. [TODO: Maybe
% phrase it to say that the assembly interface does not guarantee
% anything in terms of performance and/or energy].
% 
%               In light of this, toolchains will become more and more
% unable to statically partition and schedule code to efficiently
% utilize future parallel resources. Hardware characteristics will vary
% dynamically and we therefore cannot do without a dose of dynamism in
% the programming model: dynamic task based programming models need to
% be a part of the solution.
%             }
%           \item{
%               Task base programming models, however, rely on the
% computation and data being split into chunks (``tasks'' for the
% computation which implies a size of the data these tasks operate
% on). The size of these chunks (the granularity) is difficult to
% determine as it needs to balance the overheads of the task-based
% system with the need to expose sufficient parallelism to fully occupy
% the ever increasing number of parallel resources. A static granularity
% will be sub-optimal for future machines [TODO: Do I need to add more
% to this]
%             }
%           \end{itemize}
% % OLD STUFF
% 		  % \begin{itemize}
% 		  % \item Static partitioning does not work for everything; we need to
% 		  %     have a dynamic component.
% 		  % \item A perfectly predictable hardware does not exist and, in the
% 		  %     future will be even more far-fetched; we therefore cannot get
% 		  %     away from some level of dynamism and task-based
% 		  %     programming-models can make better use of this dynamism.
% 		  % \item Granularity is a key issue that needs to be addressed with
% 		  %     respect to locality. There is agreement that a static
% 		  %     partitioning/granularity will not work for future
% 		  %     machines. Granularity needs to balance overheads and exposing
% 		  %     parallelism.
% 		  % \end{itemize}
% 
% 
% 
% 	\item AREAS OF DISAGREEMENT (\textbf{Romain})
%           \begin{itemize}
%             \item{
%                 There is agreement on the fact that a standardization
% needs to happen for task based programming model but there is
% disagreement as to the level at which this standardization needs to
% happen. One option is to standardize the APIs at the runtime level in
% a way similar to the Open Community Runtime (OCR). Another option is
% to standardize what can be expressed by the user at the programming
% model level [TODO: Jesus, I would like to understand your position a
% bit more on this so I can clarify things a bit more].
%               }
%             \item{
%                 Another area of disagreement is how best to deal with
%  the expression of granularity; specifically, is it better for the
%  programmer to break-down tasks and data and have a runtime system
%  re-assemble them if needed or is it preferable to have the programmer
% express coarse grained tasks and data and allow the runtime system to
% break them down further. The latter approach has been used
% successfully in runtimes targeted to problems that are recursively
% divisible. The former approach would require a sort of ``recipe'' for
% the runtime to be able to stich smaller tasks or chunks of data into
% larger ones. There is debate as to which approach is simpler and more
% likely to yield positive results.
%               }
%             \end{itemize}
% 
% % OLD STUFF
% 		  % \begin{itemize}
% 		  % \item The level at which standardization should be done: task-based
% 		  %     runtime API, languages, interface to hardware, ...
% 		  % \item What/who should have control of the granularity and how should
% 		  %     it be expressed (is it easier to break-down or recompose?)
% 		  % \end{itemize}
% 
% 
% 	\item Identify Gaps / What is missing  (\textbf{Miquel})
% 	    \begin{itemize}
% 		\item Better understanding of the performance features of task-parallel
% 		systems. Need to convince developers of the value of these systems.
% 		\begin{itemize}
% 		  \item Better performance tools. Better understanding of the impact of locality
% 		  \item Case studies that show performance benefit in certain conditions (eg,
% 		    in the presence of faults?). Focus on data aspects, i.e. show that
% 		    a good management can be done even when there are a lot of work steals.
% 	      \end{itemize}
% 	  \end{itemize}
% 
% 
% 	\item APPLICATION DOMAINS  (\textbf{Hatem})
% 		\begin{itemize}
% 			\item Dense linear algebra
% 			\item Computational astronomy
% 			\item Uncertainty Quantification
% 			\item Seismic
% 			\item Weather/Climate modeling
% 			\item Krylov-based solvers
% 			\item Nbody problems
% 		\end{itemize}
% 
% \end{itemize}

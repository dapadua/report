\chapter{Data Structures and Layout Abstractions}
\label{ch:datastructures}


%Define your Area
%  Create definition of your research area
%  Describe key concepts that define your area or that were uncovered during the course of conversation
%  A few examples of work in that area (can refer back to talks on website, but no need to recount entire talk)

%Findings:
%■ Describe points/observations/discoveries/challenges/issues uncovered in the session
 %● Distill into summary (major discoveries)
 %● can refer back to presentations for details
 %● Can also use data from panel discussions
%■ Identify areas of agreement
 %● Common approaches
 %● Common concerns
%■ Identify areas of disagreement
 %● what is the substantive cause of the disagreement (document)
 %● What metrics/information/research are needed to compare/resolve
%■ Identify Gaps
 %● What is missing?

%Recommendations
%■ Opportunities for standardization of mature technologies where the is substantial agreement or commonality
 %● Have we met the necessary conditions for standardization (is the area well enough understood, are the elements of existing implementations sufficiently similar, are the benefits clearly demonstrated, is there a user community?)
 %● What should we standardize? ( Low­hanging fruit )
 %● How can we influence standards committees? (e.g. C++17standards committee?)
%■ Define research agenda for new ideas or areas where there is insufficient information to choose a final implementation ( What areas need more research?)
 %● identify research thrust
 %● what are the opportunities
 %● what needs to be done
 %● What needs to be prioritized?
 %● What resources would be required (estimate size/complexity ofthe problem if you can)
%■ How do we create a user community? (bonus question)

HIGHLIGHTS
\begin{itemize}
\item SCOPE
  \begin{itemize}
  \item In this chapter, we discuss the data locality support through data structures and data layout in a program. 
  \item Key concepts are data layout, data decomposition, data distribution, iteration space traversal (need to clearly define terminology)
  \item (im)mature solutions implemented in different langauges: Kokkos, TiDA, OpenMP extensions, GridTools, Dash, Array Extensions
  \end{itemize}

\item TERMINOLOGY (define each of these up front)
  \begin{itemize}
  \item control space(use execution space?)
  \item binding  (or execution policy ?)
  \item memory spaces
  \item data layout
  \item polymorphic data layout
  \item data decomposition 
  \item data distribution 
  \item iteration space traversal (avoid using loop traversal, maybe use domain traversal?) 
  \item access type
  \end{itemize}


\item AGREEMENTS
  \begin{itemize}
  \item Abstractions for performance portability are needed: Data layout, tile sizes, memory access patterns need to be tuned when application is moved between machines. 
  \item Changes to tunable parameters should be invisible to the programmer (either minimal code modification or no modification at all)
  \item separation of concerns: separate logical from physical 
  \item have a data model to assess the data locality: we have a model for parallelism but do not have a model for data locality
  \item easier to standardize high level concepts 
  \item low hanging fruit: 
    1) multidimensional array support (in C and C++) 
    2) polymorphic data layout: 
    ideally change the layout in runtime, 
    alternatively at compile time without requiring a lot of code modifications, 
    support layout in the type system 
  \item declaration about data vs declaration about control (I didn't understand this)
  \item there is a pragmatic reason to use C++ metaprogramming
  \item lambdas for domain traversal (or iteration space traversal)
  \end{itemize}
  
\item  DISAGREEMENTS
  \begin{itemize}
  \item Binding 
  \item support for memory spaces: can be hidden from the programmer or exposed 
  \end{itemize}

\item GAPS (what is missing? not covered at the workshop)
  \begin{itemize}
  \item a data model for which data layout is more suitable for which algorithm? or metric for locality
  \end{itemize}

\item RESEARCH AGENDA

\end{itemize}


